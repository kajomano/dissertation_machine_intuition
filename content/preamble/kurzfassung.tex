\thispagestyle{empty}
\vspace*{6\baselineskip}

\glsresetall

{\large\textbf{Kurzfassung:} }
\par

Die Mobilfunknetze der Zukunft versprechen die verbreitete Verfügbarkeit einer zuverlässigen Kommunikation mit niedriger Latenz und hoher Bandbreite, die nur durch kognitive Automatisierung erreicht werden kann.
\acp{CAN} zielen darauf ab, hochintelligente Deep-Learning-Algorithmen als Grundlage der kognitiven Funktionen zu verwenden, die ein breites Spektrum an Informationen verarbeiten und das Netz an den sich ändernden Kontext anpassen können.

Deep-Learning-Algorithmen sind führende Modellierungswerkzeuge, die in der Lage sind einige der komplexesten kognitiven Prozesse des menschlichen Denkens zu automatisieren.
Deep Learning wird jedoch überwiegend in anderen technologischen Bereichen weiterentwickelt, wobei dort der Schwerpunkt auf überwachtem Lernen liegt, das aufgrund des Mangels an zugänglichen, vorklassifizierten Daten in mobilen Netzen nur von begrenztem Nutzen ist.
\acp{CAN} erfordern Algorithmen, die auf unüberwachte Weise lernen und aus Daten Erkenntnisse ziehen können, ohne dass umfangreiche menschliche Überwachung erforderlich ist.
Diese Aufgaben erfordern eine intuitive Entscheidungsfindung mit starken kognitiven Fähigkeiten, die die Extraktion von latenten Informationen aus den Daten einschließt, und als maschinelle Intuition bezeichnet werden kann.

In dieser Dissertation werden solche Algorithmen der maschinellen Intuition, und ihre Anwendung im Kontext von Anwendungsfälle der kognitiven Netzwerkautomatisierung diskutiert.
Maschinelle Intuition wird in vier Prozesse unterteilt: Erleichterung der Kommunikation durch Exemplifizierung, erweiterte Datenetikettierung durch assoziative Modellierung, präventive Funksteuerung durch Vorhersagemechanismen und erhöhte Robustheit gegenüber verfälschten Daten durch maschinelles Vertrauen.
Logisch ist die Dissertation in vier Teile gegliedert, die den vier Prozessen entsprechen.
In jedem Teil werden eine oder mehrere Deep-Learning-basierte Implementierungen und damit verbundene Evaluierungen für einen Anwendungsfall der kognitiven Netzwerkautomatisierung vorgestellt.
Auf der Grundlage dieser Evaluierungen werden Schlussfolgerungen zu jedem intuitiven Prozess gezogen, wobei die Machbarkeit der Implementierung mit Deep-Learning-Algorithmen, die Praktikabilität in mobilen Netzwerken und die Anwendbarkeit für die kognitive Netzwerkautomatisierung bewertet werden.

Die Arbeit schließt mit einer allgemeinen Bewertung der maschinellen Intuition als Ganzes, wobei die wahrgenommene kognitive Leistung der Algorithmen, die diese Prozesse realisieren, erörtert wird, die Vorteile und Mängel des unüberwachten Lernens im Rahmen der Automatisierung von Mobilfunknetzen hervorgehoben werden und die kurz- und langfristige Zukunft der maschinellen Intuition diskutiert wird.
Abschließend weise ich noch einmal auf die Vorbehalte bei der Verwendung von Deep-Learning-Modellen in mobilen Netzwerken hin und argumentiere für eine gezielte Deep-Learning-Forschung und bessere wissenschaftliche Praktiken bei der Automatisierung mobiler Netzwerke.

\clearpage




