\thispagestyle{empty}
\vspace*{6\baselineskip}

{\large\textbf{Abstract:} }
\par

The mobile networks of the future promise the widespread availability of reliable, low-latency and high-bandwidth communication, which can only be achieved through cognitive automation.
\acp{CAN} aim to use highly intelligent deep learning algorithms as the core of cognitive functions, capable of processing a wide scope of information, and use it to adapt the network to changing context.

Deep learning algorithms are the state-of-the-art modeling tools, capable of automating some of the most complex cognitive processes of human reasoning.
However, deep learning is mostly advanced in other fields, with a focus on supervised learning, which is of limited use in mobile networks on account of a lack of easily accessible labeled data.
\acp{CAN} require algorithms that are capable of learning in an unsupervised manner, inferring insight from data without the help of extensive human supervision.
These tasks necessitate intuitive decision making with strong cognitive capabilities, involving the extraction of hidden, latent information from the data, the set of capabilities which can be termed as machine intuition.

This thesis discusses such machine intuition algorithms, and their application to cognitive network automation use cases.
Machine intuition is divided into four processes: facilitation of communication through exemplification, augmented labeling through associative modeling, preemptive radio control through prediction, and increased robustness against corrupted inputs through machine confidence.
The core of the thesis is split into four parts corresponding to the four processes, with each presenting one or more deep-learning-based implementations, and related evaluations on a cognitive network automation use case.
Based on these evaluations, conclusions about each intuitive process is given, assessing its feasibility of implementation with deep learning algorithms, its practicality in mobile networks, and its applicability to cognitive network automation.

The thesis is concluded in a general assessment of machine intuition as a whole, discussing the perceived cognitive power the algorithms realizing these processes, highlighting the advantages and shortcomings of unsupervised learning in the mobile network automation setting, and predicting short- and long-term future of machine intuition.
As closing remarks, I reiterate the caveats of using deep learning models in mobile networks, and argue for dedicated deep learning research and better scientific practices in mobile network automation.

\clearpage
